\documentclass[autodetect-engine,dvipdfmx-if-dvi,ja=standard, 12pt]{bxjsarticle}

% 二段組にするとき
% \documentclass[twocolumn,autodetect-engine,dvipdfmx-if-dvi,ja=standard]{bxjsarticle}

\usepackage{graphicx}        %図を表示するのに必要
\usepackage{color}           %jpgなどを表示するのに必要
\usepackage{amsmath,amssymb} %数学記号を出すのに必要
\usepackage{setspace}
\usepackage{cases}
\usepackage{here}
\usepackage{fancyhdr}
\usepackage{ascmac}
\usepackage{url}


\setlength{\textheight}{\paperheight}   % 紙面縦幅を本文領域にする(BOTTOM=-TOP)
\setlength{\topmargin}{3truemm}       % 上の余白を30mm(=1inch+4.6mm)に
\addtolength{\topmargin}{-\headheight}  %
\addtolength{\topmargin}{-\headsep}     % ヘッダの分だけ本文領域を移動させる
\addtolength{\textheight}{-50truemm}    % 下の余白も30mm(BOTTOM=-TOPだから+TOP+30mm)
% #################### Landscape Setting #######################
% # LEFT = 1inch + \hoffset + \oddsidemargin (\evensidemargin) #
% #      = 1inch + 0pt + 0pt                                   #
% # RIGHT = \paperwidth - LEFT - \textwidth                    #
% ##############################################################
\setlength{\textwidth}{\paperwidth}     % 紙面横幅を本文領域にする(RIGHT=-LEFT)
\setlength{\oddsidemargin}{-5.4truemm}  % 左の余白を25mm(=1inch-0.4mm)に
\setlength{\evensidemargin}{-5.4truemm} %
\addtolength{\textwidth}{-40truemm}     % 右の余白も25mm(RIGHT=-LEFT)
% 行頭の字下げをしない
\parindent = 0pt

% ヘッダとフッタの設定
\lhead{ネットワークシステム}
\chead{課題6}
\rhead{5E 20番 佐藤凌雅}
\lfoot{}
\cfoot{-\thepage-} % ページ数
\rfoot{}

% 式の番号を(senction_num.num)のようにする
\makeatletter
\@addtoreset{equation}{section}
\def\theequation{\thesection.\arabic{equation}}
\makeatother

% 画像の貼り付けを簡単にする
\newcommand{\pic}[2]
{
  \begin{figure}[H]
    \begin{center}
      \includegraphics[scale=#2]{#1}
    \end{center}
  \end{figure}
}

% 単位の記述を簡単にする
\newcommand{\unit}[1]
{
  \, [\mathrm{#1}]
}
\begin{document}
\pagestyle{fancy}

\subsection{課題内容}
 著作権等侵害罪を一部非親告罪化する法律が2018年7月6日に公布(平成30年法律第70号)され,2018年12月30日に施行されました.これに関連して,以下の課題についてレポートを提出してください.\\
【参考URL】\url{http://www.bunka.go.jp/seisaku/chosakuken/hokaisei/kantaiheiyo_hokaisei/}

\subsubsection{著作権法はインターネットを介して確認することができます.そのURLを書いてください.}
\url{http://www.cric.or.jp/db/domestic/a1_index.html}

\subsubsection{著作権法の中から目的について記載された条文を記載してください.}
著作権法 第一章 総則 第一節 通則 (目的) 第一条\\
 この法律は、著作物並びに実演、レコード、放送及び有線放送に関し著作者の権利及びこれに隣接する権利を定め、これらの文化的所産の公正な利用に留意しつつ、著作者等の権利の保護を図り、もつて文化の発展に寄与することを目的とする。\\
(昭六一法六四・一部改正)

\subsubsection{改正後の第123条第2項及び第3項について概要を説明してください.}
 著作権等侵害罪の一部非親告罪化するもの.\\
 改正前の著作権法においては,著作権等を侵害する行為は刑事罰の対象となるものの,これらの罪は親告罪とされており,著作権者等の告訴がなければ公訴を提起することができなかった.\\
 改正により,著作権等侵害罪のうち,条文に記載されている全ての要件に該当する場合に限り,非親告罪とし,著作権等の告訴がなくとも公訴を提起することができるというもの.

\subsubsection{著作権に関して,非親告罪が適用されるケースと親告罪が適用されるケースについて例をあげて説明してください.}
非親告罪が適用されるケース\\
販売中の漫画や小説の海賊版を販売する行為や,映画の海賊版をネット配信する行為等を権利者の告訴がなくても公訴提起できる.\\

親告罪が適用されるケース\\
著作権侵害が発生した際,権利者のみが公訴提起できる.

\subsubsection{フェアユースについて説明してください.}
【参考URL】https://ja.wikipedia.org/wiki/フェアユース\\
フェアユース とは、アメリカ合衆国の著作権法などが認める著作権侵害の主張に対する抗弁事由の一つ。同国の著作権法107条によれば,著作権者の許諾なく著作物を利用しても,その利用が4つの判断基準のもとで公正な利用に該当するものと評価されれば,その利用行為は著作権の侵害にあたらない.

\subsubsection{著作権等侵害罪の一部非親告罪化の是非について意見を述べよ.}
【参考URL】https://ja.wikipedia.org/wiki/日本の著作権法における非親告罪化\\
【参考】意見そのものには「正しい」/「間違っている」はない(つまりどんな意見を主張しようが自由).ただしそれをサポートするファクトには「正しい」/「間違っている」がある\\

私は,非親告罪化には反対である.\\
理由は,二次創作などのジャンルの萎縮に繋がってしまうと思うからである.現在の著作権法では,二次創作物の非親告罪で訴えられる可能性は低い.しかし,仮にこれからの法改正で権利者でないものが著作権侵害で訴えるようになれば,二次創作を行う人も少なくなってしまうと考える.あるコンテンツを知るとき,必ずしも公式(権利者)の発表したものでないから知ることは少なくない.二次創作という,作品を知る機会を縮小することは市場規模を小さくしていき,最終的には大元のコンテンツを製作する人たちにも影響を与えてしまうと考える.


\end{document}
